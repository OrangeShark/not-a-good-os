\chapter{Bootloader and Loading the Kernel}

\section{What is a bootloader?}

The bootloader is a simple application that is used to initialize an environment
for a kernel to run. Some boot loader tasks involve probing the system for
things such as video modes and the amount of available system memory. The
bootloader gets values for these and hands this information over to the kernel
so that it can initialize properly.

\section{Bootloader Limitations}

Due to BIOS limitations the boot sector (where the BIOS will try to start
executing code) is limited to a single drive sector of 512 bytes even on drives
where the sector size is larger than 512 bytes. For this reason newer
bootloaders that have exceeded this size have to resort to some drastic measures
in order to actually boot. To do this they separate the bootloader into
different stages where the first stage fits in this 512 byte limit and it's only
purpose is to load the rest of the bootloader into memory and continue
execution.

When your bootloader is called into by the BIOS it starts in the processor's
``Real Mode''. In this mode the current execution context on the CPU (in this
case the bootloader) has access to BIOS interrupts such as reading/writing to
disk and setting the system clock.

\section{Loading A Kernel}

In order to load the kernel the bootloader just needs to know a couple of
things: what disk the kernel is located on and where on the disk the kernel is
at. Once it knows these two things it can use BIOS interrupts to start reading
the kernel binaries from the disk.

\subsection{Kernels larger than 1MB}

Due to a restriction of Real Mode you only have access to 20 bits of segmented
memory space (1MB). This means that you have to be a bit more clever to load
kernels that are larger than this into memory.

To get around this you can have the bootloader load only your real mode kernel
code into the region which will then act as a bootstrapper to load the rest of
the kernel and switch to protected mode.

\section{Switching to Protected Mode}

\subsection{Enabling the A20 Line}

The A20 line refers to address line 20 or the 21st bit used for addressing
memory on the address bus. Enabling this enables access to the High Memory Area
(HMA) which resides just beyond the 1MB limit of real mode.
